\documentclass{beamer}
\usetheme{Boadilla}

\usepackage{amsmath}
\usepackage{array}
\usepackage{amsfonts}
\usepackage{hyperref}
\usepackage{biblatex}
\graphicspath{ {./images/} }


\title{Methods comparison for graphs}
\author{Ksenofontov Gregory}
\institute{MIPT}


\begin{document}

\begin{frame}
    \titlepage
\end{frame}

\begin{frame}{Main idea of Riemannian approach}
        Let we have EEG trial $X\in\mathbb{R}^{C\times T}$, a record of  $C$  electrodes with $T$ time samples. And the record follows multivariate normal model, i.e. $X \sim \mathcal{N}(0, \Sigma)$, where $\Sigma\in \mathbb{R}^{C\times C}$. \par
        Information theory gives us metric, the Fisher information, for probability distributions. As probability distributions belong on Riemann manifold, with such metric we can define Riemann distance $\delta_R$ between distributions. Let we have $X_1 \sim \mathcal{N}(0, \Sigma_1)$ and $X_2 \sim \mathcal{N}(0, \Sigma_2)$ the distance would be:
        \begin{equation}
            \delta_R(\Sigma_1, \Sigma_2)=||\log(\Sigma_1^{-1/2}\Sigma_2\Sigma_1^{-1/2})||_F = \left[ \sum_{c=1}^C\log^2\lambda_c\right]^{1/2},
    \end{equation}
    where $\lambda_c$ - all eigenvalues of $\Sigma_1^{-1/2}\Sigma_2\Sigma_1^{-1/2}$
\end{frame}
\begin{frame}{Minimum Distance to Riemannian Mean (MDM)}
    Let we have dataset $X_i \sim \mathcal{N}(0, \Sigma_i)$ and corresponding labels $y_i\in\{1:T_c\}$. The training process is computing the covariance matrix for each  $T_c$ classes using Riemann geometric mean:
    \begin{equation}
      \overline{\Sigma}_K = \mathcal{G}(\Sigma_i|y_i=K) = \arg\min_{\Sigma\in P(n)}\sum_i\delta_R^2(\Sigma, \Sigma_i),
  \end{equation}
  where $K\in [1:T_c]$ - class label \par
  The inference is finding of minimum distance, i.e.
  \begin{equation}
   \hat y = \arg\min_K\delta_R(\overline{\Sigma}_K, \Sigma)   
 \end{equation}
 Other methods\footnote{\href{https://arxiv.org/pdf/1409.0107.pdf}{A Plug\&Play P300 BCI Using Information Geometry}} introduces more complex representation of covariance matrices. 
\end{frame}
\begin{frame}{Comparison}
\begin{table}[]
    \centering
    \begin{tabular}{ | c | c | } 
     \hline
         Strengths & Weaknesses \\ 
     \hline
     \multicolumn{2}{| c |}{Riemannian approach}\\
     \hline
         Less data to learn  &  Complexity depends on $C$  \\ 
         Easy to understand & No connection between electrodes  \\ 
           & Only classification? \\ 
     \hline
     \multicolumn{2}{| c |}{Graph Laplacian approach}\\
     \hline
         Uses connection  & Uses diff equations  \\ 
          & Complex inference  \\
     \hline
    \end{tabular}
    \caption{Comparison of approaches}
\end{table}
    
\end{frame}
\end{document}